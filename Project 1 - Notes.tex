\documentclass[a4paper]{article}
\usepackage{fancyhdr}
\usepackage{lastpage}
\usepackage[utf8]{inputenc}
\usepackage[official]{eurosym}
\usepackage[left=2cm, right=2cm, top=2cm]{geometry} %right=30mm,left=30mm,top=15mm,bottom=20mm für die Bachelorarbeit/Seminararbeit
\usepackage{graphicx} % support the \includegraphics command and options
\usepackage{amsmath}
\usepackage{amssymb}
\usepackage{adjustbox}
\usepackage{mathtools}
\usepackage{centernot}
\usepackage{sectsty}
\usepackage[parfill]{parskip} % Activate to begin paragraphs with an empty line rather than an indent
\usepackage{booktabs} % for much better looking tables
\usepackage{array} % for better arrays (eg matrices) in maths
\usepackage{paralist}% very flexible & customisable lists (eg. enumerate/itemize, etc.)
\usepackage{verbatim} % adds environment for commenting out blocks of text & for better verbatim
\usepackage{subfig} % make it possible to include more than one captioned figure/table in a single float
\usepackage[table,xcdraw]{xcolor}
\usepackage{perpage}
\usepackage{hyperref}
\usepackage[perpage,symbol*]{footmisc}
\usepackage{chngcntr}
\usepackage{amsthm}
\usepackage{makecell}
\usepackage{csquotes}
\usepackage{lastpage}
\usepackage{tikz}
\usepackage{pgfplots}
\usepackage{framed}
\usepgfplotslibrary{fillbetween}
\usetikzlibrary{patterns}
\subsectionfont{\itshape}
\makeatletter
\renewcommand{\@seccntformat}[1]{}
\makeatother
\allowdisplaybreaks
\linespread{1.1}
\pagestyle{fancy}
\fancyhf{}
\lhead{Project I: One-Sector Growth Model} % add others
\rhead{Page \thepage \hspace{1pt} of \pageref{LastPage}}
\chead{Project I}
\title{Dynamic Macroeconomics with Numerics: Project I}
\author{Hashem Zehi, Samuel (120112285)\\Kotiers, Róza (11945569)\\Polzin, Julian (11948952)}
\date{\today}
\theoremstyle{definition}
\newtheorem{definition}{Definition}[section]
\newtheorem{lemma}{Lemma}[section]
\newtheorem{exmp}{Example}[section]
\newtheorem{note}{Note}[section]
\newtheorem{prop}{Proposition}[section]
\newcommand\Tau{\mathcal{T}}
\renewcommand{\thefootnote}{\fnsymbol{footnote}}
\newcommand*\diff{\mathop{}\!\mathrm{d}} %Integral d 
\newcommand\independent{\protect\mathpalette{\protect\independenT}{\perp}}
\def\independenT#1#2{\mathrel{\rlap{$#1#2$}\mkern2mu{#1#2}}}
\begin{document}
\maketitle
\newpage
\section{Part I: Deterministic Version}
We have a representative agents with preference given by
	\begin{align*}
	\sum\limits_{t=0}^{\infty} \beta^t u(c_t),
	\end{align*}
with $\beta \in (0,1)$. We assume $\partial u(c_t)/\partial c_t > 0$, and $u$ being concave and twice differentiable. Output is now a function of the utilized machines $\kappa_t$. This gives the aggregate resource constraint
	\begin{align*}
	c_t + i_t \leq z f(\kappa_t),
	\end{align*}	
with all variables having their usual meaning. $f$ is strictly increasing, concave and twice differentiable. Furthermore, the marginal product of capital converges to zero for the stock going to infinity. Unused capital does not depreciate. We have capital accumulation according to 
	\begin{align*}
	k_{t+1} \leq (1-\delta)\kappa_t + (k_t - \kappa_t)+i_t,
	\end{align*}	
where we have $k_t \geq \kappa_t$. 
\subsection{(a)-(b) social planner problem and steady state description}
The social planner problem is written as the Lagrangian
	\begin{align*}
	\mathfrak L(\dots) = \sum\limits_{t=0}^{\infty} \beta^t \Bigg\{ u(c_t) + \lambda_t \Big(z f(\kappa_t)- k_{t+1}+(1-\delta)\kappa_t+k_t - \kappa_t -c_t \Big)+ \theta_t \Big( k_t - \kappa_t \Big) \Bigg\}
	\end{align*}
The $[\dots]$ denote the decision variables and the Lagrangian multipliers, i.e.\ consumption in the current period $c_t$, next periods capital stock $k_{t+1}$, and current period's capital utilization $\kappa_t$ in addition to the multipliers $\theta_t,\ \lambda_t$. 
	
Then we have the FONCs, and first start with the one for consumption:
	\begin{align}
	\frac{\partial}{\partial c_t}\mathfrak L(\dots) 			&= \beta^t \Big\{ u^\prime(c_t)-\lambda_t \Big\} \overset{!}{=}0, \\
	\frac{\partial}{\partial \kappa_t}\mathfrak L(\dots)		&= \beta^t \Big\{ \lambda_t( z f^\prime(\kappa_t)+(1-\delta)-1)-\theta_t \Big\} \overset{!}{=} 0, \\
	\frac{\partial}{\partial k_{t+1}}\mathfrak L(\dots)		&= \beta^t \Big\{ \lambda_t(-1) \Big\} + \beta^{t+1} \Big\{ \lambda_{t+1}+\theta_{t+1} \Big\} \overset{!}{=}0.  \\
	\frac{\partial}{\partial \lambda_t}\mathfrak L(\dots)	&= \beta^t \Big( z f(\kappa_t)- k_{t+1}+(1-\delta)\kappa_t+k_t - \kappa_t -c_t \Big) \overset{!}{=} 0 \nonumber \\
	\frac{\partial}{\partial \theta_t}\mathfrak L(\dots)		&= \beta^t (k_t - \kappa_t) \overset{!}{=} 0. \nonumber
	\end{align}
Note that we also have the derivatives w.r.t.\ the Lagrangian multipliers. These are not discussed at the beginning but the multipliers are taken into consideration at a later point where we briefly discuss whether or not certain constraints are binding, with particular focus on the capacity constraint.
	
From the first FONC we have
	\begin{align*}
	u^\prime(c_t) = \lambda_t.
	\end{align*}
The shadow price on this constraint is equal to the marginal utility of consumption in each period. From the second equation we have
	\begin{align*}
	\theta_t = \lambda_t\big( z f^\prime(\kappa_t)-\delta \big)
	\end{align*}
This is the shadow price on the capacity constraint is equal to the marginal utility from consumption multiplied by the marginal product of utilized capital minus the depreciation rate. Note that this implies that if the marginal product of utilized capital is equal to the depreciation rate, the shadow price on the constraint must be zero, regardless of the consumption level in the respective period.
	
From the third FONC we have
	\begin{align*}
	\lambda_{t} = \beta \Big(\lambda_{t+1} + \theta_{t+1}  \Big)
	\end{align*}	
and using the previous results we have
	\begin{align*}
	u^\prime(c_t) 	&= \beta u^\prime(c_{t+1})+ \beta u^\prime(c_{t+1})(z f^\prime(\kappa_{t+1})-\delta)
	\intertext{which can be written as}
	u^\prime(c_t)	&= \beta u^\prime(c_{t+1})[zf^\prime (\kappa_{t+1})+1-\delta], \\
	1 					&= \beta \frac{u^\prime(c_{t+1})}{u^\prime(c_t)}[z f^\prime (\kappa_{t+1})+1-\delta)].
	\end{align*}		
which is the Euler equation. Economically, it represents the value of an additional marginal unit of consumption in two consecutive periods. Giving up one marginal unit of consumption in the current period translates into one additional marginal unit of capital that can be used in the next period. This additional capital yields the return of $z f^\prime(\kappa_{t+1})-\delta$. Therefore, it is optimal to equalize the marginal utility of consumption of period t with the discounted sum of marginal utility of consumption in period $t+1$ and the marginal utility from consumption generated by the return on that investment.

For the conditions to be sufficient we additionally impose the transversality condition
	\begin{align*}
	\lim\limits_{t\rightarrow\infty} \beta^{t-1} u^\prime(c_{t-1})k_t = 0.
	\end{align*}	
The transversality condition can be seen as a terminal condition, although we are working with an infinite time horizon. It could be seen as a guarantee that the household does not hold any positive or negative capital stock at the end of time, since it does not yield any utility. This means that the household would not accumulate capital for the end of its lifetime, as any leftover capital would need to have economic value of zero.
		
Assume a binding utilization constraint where we then have $\kappa_t = k_t$ and $\theta_t \neq 0$. The inter-temporal rate of substitution must be smaller 1. Now assume it does not bind, i.e.\ $\theta_t = 0$, $k_t > \kappa_t$, then we have the inter-temporal rate of substitution being equal one, $z f^\prime (\kappa_t) = \delta$. This implies that we make use of the machines up until the point where this equation holds. Assuming zero investment, then the marginal product of utilization is lower than the depreciation rate, meaning that the capital stock is too high and will reduce.

We need to argue why we have full capacity usage in the steady state. We had from the third FONC that in the steady state we must have $\lambda = \beta (\lambda+\theta)$. The capacity constraint being non-binding implies $\theta = 0$, which in turn yields
	\begin{align*}
	\lambda = \beta \lambda \Leftrightarrow \beta = 1,
	\end{align*}
which is a contradiction, thus the constraint must be binding and $\kappa_t = k_t$. Very informal, but this result is not particularly surprising. We can view this non-utilization as some kind of saving mechanism that has a marginal product of zero. But if we do utilize this capital then we have a strictly positive marginal product, so there is no incentive to not fully utilize this capital within the given model framework.

Now, in order to ensure a unique steady state, we have the following. First consider the FOCs where have in the steady state that
	\begin{align*}
	1 = \beta (z f^\prime(k^*)+1-\delta),
	\end{align*}
and also the consumption level being given by the output minus the investment, which equals the depreciated capital in the steady-state:
	\begin{align*}
	c^* = z f(k^*) - \delta k^*.
	\end{align*}	
As we have $f$ satisfying the Inada conditions and being monotone. The Inada conditions are given by
	\begin{align*}
	\lim\limits_{k\rightarrow\infty}f^\prime(k)=0,\ \lim\limits_{k\rightarrow0} f^\prime(k) = \infty.
	\end{align*}
The Inada conditions ensure a reasonable economic view on the value of capital for the production function. With a capital stock that is practically non-existent, capital becomes very valuable, since it is essential for production. Therefore, it is optimal to have a positive capital stock under usual conditions. Additionally, capital has a positive but decreasing marginal product, which eventually approaches zero if it is increased infinitely. This ensures that it is not optimal for a firm to try to approach a capital stock as large as possible, which would not be intuitive.
\subsection{(c) steady state capital utilization}
Assuming the steady state with full capital utilization, then we have the following:
	\begin{align*}
	\kappa_t 					&\rightarrow \kappa^* \\
	k_t ,k_{t+1}				&\rightarrow k^* \\
	c_t 						&\rightarrow c^* \\
	i_t 							&\rightarrow i^* 
	\end{align*}
and of course $\kappa^* = k^*$. We look at the Euler equation, and then we have
	\begin{align*}
	1 							&= \beta \big[ z f^\prime(\kappa^*)+1-\delta \big] \\
	\beta^{-1} +\delta-1	&= z f^\prime (\kappa^*) \\
	\kappa^* 				&= f^{\prime -1} \Big( \frac{\beta^{-1} + \delta - 1}{z} \Big).
	\end{align*}	
This is the steady state level of capital utilization, a function only of the parameters of the model and the functional form of the derivative of the production function w.r.t.\ the capital utilization.	
%
%
%
%
%
%
%
%
%
%
%
%
%
\newpage
\section{Part II: Stochastic One-Sector Growth Model with Variable Capital Utilization}
The preferences of the \textbf{representative agent} are given by
	\begin{align*}
	\sum\limits_{t=0}^{\infty} \beta^t \log(c_t) ,
	\end{align*} 
with the production function given by
	\begin{align*}
	y_t = z_t^{1-\alpha}(k_t U_t)^\alpha,
	\end{align*}	
with $\alpha,\beta,U_t \in (0,1)$. Using capital more intensely increases the capital depreciation rate
	\begin{align*}
	\delta_t = \delta U_t^\phi,
	\end{align*}	
given $\delta \in (0,1)$ and $\phi > 1$. The law of motion for physical capital is given by
	\begin{align*}
	k_{t+1} = (1-\delta_t)k_t + i_t.
	\end{align*}	
Technological progress is captured by
	\begin{align*}
	z_t 	&= \exp (x_t) \\
	x_t 	&= \rho x_{t-1} + \epsilon_t,
	\end{align*}	
with $\rho \in (0,1)$, such that we have non-explosive behavior to shocks on the error term $\epsilon_t \overset{\text{i.i.d.}}{\sim} \mathcal N(0,\sigma_\epsilon)$. In the stationary steady-state we have $\sigma_\epsilon = 0$. The aggregate resource constraint is given by
	\begin{align*}
	c_t + i_t \leq y_t.
	\end{align*}
\subsection{(a): Social planner problem}	
Since we have a stochastic term need to formulate the maximization problem in terms of expectation formed in period zero:
	\begin{align*}
	\max\limits_{c_t} \mathbb E_0 \sum\limits_{t=0}^{\infty} \beta^t \log (c_t),
	\end{align*}
subject to the conditions laid out above. We can consolidate these conditions as follows; first use the production function in the aggregate resource constraint such that we have
	\begin{align*}
	c_t \leq z_t^{1-\alpha}(k_t U_t)^\alpha - i_t.
	\end{align*}	
We then rewrite the law of motion for physical capital keeping the gross investment on one side such that
	\begin{align*}
	i_t = k_{t+1} - (1-\delta_t)k_t = k_{t+1} - k_t + \delta_t k_t .
	\end{align*}	
Now we use the capital depreciation rate and have
	\begin{align*}
	i_t = k_{t+1} - k_t + \delta U_t^\phi k_t. 
	\end{align*}	
We can now use this in the reformulation of the aggregate resource constraint:
	\begin{align*}
	c_t \leq z_t^{1-\alpha}(k_t U_t)^\alpha - k_{t+1} + k_t - \delta U_t^\phi k_t.
	\end{align*}	
This is then plugged into the original maximization.
	\begin{leftbar}
		\begin{prop}[Social Planner Problem]
		The social planner maximizes the expected discounted lifetime utility with respect to current period capital utilization and next period's capital stock:
			\begin{align}
			\max\limits_{U_t,k_{t+1}} \mathbb E_0 \sum\limits_{t=0}^{\infty} \beta^t \log \Big(  z_t^{1-\alpha}(k_t U_t)^\alpha - k_{t+1} + k_t - \delta U_t^\phi k_t \Big)
			\end{align}	
		subject to the stochastic technology level which follows
			\begin{align}
			z_t = \exp(x_t),\ x_t = \rho x_{t-1} + \epsilon_t,
			\end{align}	
		given some initial capital stock $k_0 \geq 0$. 	
		\end{prop}
	\end{leftbar}	
Furthermore, we can take the partial derivative of the argument which is supposed to be maximized, equation (4) w.r.t.\ $k_{t+1}$ to obtain the first FONC
	\begin{align*}
	\mathbb E_0 \beta^t \Big\{ \frac{-1}{z_t^{1-\alpha}(k_t U_t)^\alpha - k_{t+1} + k_t - \delta U_t^\phi k_t} \Big\} + \mathbb E_0 \beta^{t+1} \Big\{ \frac{\alpha U_{t+1} z_{t+1}^{1-\alpha}(k_{t+1}U_{t+1})^{\alpha-1} + 1 - \delta U_{t+1}^\phi}{z_{t+1}^{1-\alpha}(k_{t+1}U_{t+1})^\alpha-k_{t+2}+k_{t+1}-\delta U_{t+1}^\phi k_{t+1}} \Big\} \overset{!}{=} 0 
	\end{align*}
We can write this as the Euler equation for this maximization problem
	\begin{align*}
	\mathbb E_0 \frac{1}{c_t} = \mathbb E_0 \beta \frac{1}{c_{t+1}} \frac{\partial c_{t+1}}{\partial k_{t+1}} \Leftrightarrow 1 = \mathbb E_0 \beta \frac{c_{t}}{c_{t+1}} \frac{\partial c_{t+1}}{\partial k_{t+1}}
	\end{align*}
Taking the partial derivative of the to be maximized argument in (4) w.r.t.\ $U_t$ yields the second FONC:
	\begin{align*}
	\mathbb E_0 \beta^t \frac{\alpha z_t^{1-\alpha}(k_t U_t)^{\alpha-1}k_t- \phi \delta U_t^{\phi-1}k_t}{c_t} \overset{!}{=} 0
	\end{align*}
This FONC is equivalent to the expression
	\begin{align*}
	\mathbb E_0 \phi \delta U_t^{\phi-1}k_t 	&= \mathbb E_0 \alpha z_t^{1-\alpha}(k_t U_t)^{\alpha-1}k_t \\
	\phi \delta U_t^{\phi-1}k_t 	&= \mathbb E_0 \alpha z_t^{1-\alpha} k_t^{\alpha}U_t^{\alpha-1} \\
	\mathbb E_0 \phi \delta U_t^{\phi-1} 		&=\mathbb E_0 \alpha z_t^{1-\alpha} k_t^{\alpha-1} U_t^{\alpha-1} \\
	\mathbb E_0 U_t^{\phi-\alpha} 				&=\mathbb E_0  (\phi \delta)^{-1}  \alpha z_t^{1-\alpha} k_t^{\alpha-1} \\
	\mathbb E_0 U_t 								&=\mathbb E_0  \Big\{ (\phi \delta)^{-1}  \alpha z_t^{1-\alpha} k_t^{\alpha-1} \Big\}^{\frac{1}{\phi-\alpha}}.
	\end{align*}
In words, the capital utilization only depends on the current level of physical capital, the current level of technology and the parameters of the model. 

For the description of the time paths we omit the expectation operator such that we can fully capture the stochastic nature of the model. We have the time-path for capital given by
	\begin{align}
	k_{t+1} 	&=  (1-\delta U_t^\phi)k_t - i_t, \nonumber
	\intertext{and using $i_t = y_t - c_t$ we have}
				&= (1-\delta U_t^\phi)k_t - z_t^{1-\alpha} (k_t U_t)^\alpha + c_t.
	\end{align}
As $U_t(k_t,z_t)$, this is a valid time-path for the expected next-period capital stock.  Note that if we insert the time path for capital into the capital utilization, the next-period consumption, as well as the next-period investment equations which follow, we have the next period's functions being a function of current period's capital stock, and other past values, e.g.\ we may use AR(1) process governing the TFP. We could make use of this to write $U_{t+1}$ as a function of period $t$ values, giving the proper time-path interpretation to the equations describing how investment, capital utilization, and consumption develop over time. We use this to give $U_{t+1}$ and $z_{t+1}$ as follows:
	\begin{align}
	z_{t+1} 	&= \exp(\rho x_{t}+\epsilon_{t+1}) \\
	U_{t+1}	&= \Big\{ (\phi \delta)^{-1}  \alpha \exp(\rho x_{t}+\epsilon_{t+1})^{1-\alpha} \Big( (1-\delta U_t^\phi)k_t - z_t^{1-\alpha} (k_t U_t)^\alpha + c_t \Big)^{\alpha-1} \Big\}^{\frac{1}{\phi-\alpha}}
	\end{align}
Clearly both equations are simply functions of past values, as well as the current period $\epsilon$, which gives the stochastic nature of the model. Omitting this term by only using expected values for the variables would not give an appropriate time-path for the model, but rather one for the deterministic equivalent.
	
Using the Euler equation, we can write the time-path of consumption as
	\begin{align*}
	c_{t+1} 	&= \beta  {c_{t}} \frac{\partial c_{t+1}}{\partial k_{t+1}}  \\
				&= \beta {c_t} \Big[ \alpha U_{t+1} z_{t+1}^{1-\alpha} (k_{t+1}U_{t+1})^{\alpha-1}+1-\delta U_{t+1}^\phi \Big],
	\end{align*}	
where we could again use 	(7) and (8), but omit this such that the equations are easier to read. 

The time path for gross investment is given by
	\begin{align*}
	 i_{t+1} 	&= z_{t+1}^{1-\alpha}(k_{t+1} U_{t+1})^\alpha - \beta  {c_{t}} \frac{\partial c_{t+1}}{\partial k_{t+1}} \\
				&= z_{t+1}^{1-\alpha}(k_{t+1} U_{t+1})^\alpha-\beta {c_t} \Big[ \alpha U_{t+1} z_{t+1}^{1-\alpha} (k_{t+1}U_{t+1})^{\alpha-1}+1-\delta U_{t+1}^\phi \Big]
	\end{align*}
\subsection{(b): Steady state calculation}
Starting with the level of technology, we know that $x^* \rightarrow 0$ in a world without shocks, as any shocks die out due to $\rho \in (0,1)$. Thus we have $z^* \rightarrow \exp(0) = 1$. For the capital stock, using the law of motion for physical capital, we have
	\begin{align*}
	k^* = (1-\delta^*)k^* + i^* \Leftrightarrow i^* = \delta^* k^*,
	\end{align*}
namely that the gross investment equals the depreciated capital.	

We then have the stationary steady state of capital utilization given by
	\begin{align*}
	U = \Big( \alpha(\delta\phi)^{-1} k^{\alpha-1} \Big)^{\frac{1}{\phi-\alpha}}
	\end{align*}
We may use this in the Euler equation to get
	\begin{align*}
	0 	&= \beta \Big\{ \alpha U z^{1-\alpha}(k U)^{\alpha-1} + 1 - \delta U^\phi \Big\} -1 
	\intertext{and when plugging in the previous result, dropping $z$ since it is equal to one in the stationary steady state, we have}
	0 	&= \beta \Big\{ \alpha U^\alpha k^{\alpha-1} + 1 - \delta U^\phi \Big\} - 1 \\
	0	&= \beta \Big\{ \alpha \Big( \alpha(\delta\phi)^{-1} k^{\alpha-1} \Big)^{\frac{\alpha}{\phi-\alpha}}k^{\alpha-1}+1-\delta \Big( \alpha(\delta\phi)^{-1} k^{\alpha-1} \Big)^{\frac{\phi}{\phi-\alpha}} \Big\}-1
	\end{align*}
Using MATLAB we find that
	\begin{align*}
	k^* 		&= 60.6232, \\
	U^* 		&= 0.6471,\\
	\delta^* 	&= 0.0148.
	\end{align*}
The steady state capital utilization is approximately 65\%, the steady state depreciation rate approximately 1.48\%, and the capital stock is approximately 60.62\%.





\end{document}